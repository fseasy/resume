% !TEX TS-program = xelatex
% !TEX encoding = UTF-8 Unicode
% !Mode:: "TeX:UTF-8"

\documentclass{resume}
\usepackage{zh_CN-Adobefonts_external} % Simplified Chinese Support using external fonts (./fonts/zh_CN-Adobe/)
%\usepackage{zh_CN-Adobefonts_internal} % Simplified Chinese Support using system fonts
\usepackage{linespacing_fix} % disable extra space before next section
\usepackage{cite}

\begin{document}
\pagenumbering{gobble} % suppress displaying page number

\name{徐伟}

% {E-mail}{mobilephone}{homepage}
% be careful of _ in emaill address
\contactInfo{readonlyfile@hotmail.com}{(+86) 152-1071-6440}{https://memeda.github.io}
% {E-mail}{mobilephone}
% keep the last empty braces!
%\contactInfo{xxx@yuanbin.me}{(+86) 131-221-87xxx}{}

\section{\faHistory\ 工作经历}

\datedsubsection{\textbf{百度}}{2017年7月 -- 至今}
\expattr{NLP算法工程师(T5)}{自然语言处理部}{北京}
\begin{onehalfspacing}
负责团队文本摘要子方向(共1$ \rightarrow $2人),持续迭代技术,有力支持业务
\begin{itemize}
  \item 独立设计、实现工业界领先的文本摘要系统;对内支持核心产品(Feed、小度等),对外赋能企业及个人(通过AI开放平台、智能创作平台等)
  \item 工作覆盖数据、评估、技术、落地、规划5个维度;关键积累100\%正式化,保证方向可持续发展
  \item 获部门奖项3次;指导、协同新人近1年;撰写专利4项
\end{itemize}
\end{onehalfspacing}

\section{\faUsers\ 项目经历 \small(按完成度降序)}
\datedsubsection{\textbf{面向实用的工业级中文文本摘要系统}}{2018年1月 -- 至今}
\role{项目负责人,项目指导人,策略执行者}{@百度}
\begin{onehalfspacing}
文本摘要系统是团队能力输出基石和成熟技术积淀;旧系统效果不佳难维护,需重构算法、代码
\begin{itemize}
  \item 类型分发+多摘要策略分治应对多元输入;核心模型多层次表征输入,多维度刻画句子重要性
  \item 建立原型$\rightarrow$LIB工作流,兼顾开发、运行效率;LIB 59天发版,1.5W行代码,低耦合高内聚
  \item 迭代近2年,稳定、持续支持多个重点业务;摘要效果绝大部分时间领先竞品
\end{itemize}
\end{onehalfspacing}

\datedsubsection{\textbf{摘要技术难点攻坚:连贯性优化,Low-Resource摘要生成优化}}{2019年4月 -- 至今}
\role{项目负责人,策略执行者}{@百度}
\begin{onehalfspacing} 
针对摘要在实际应用中的难点,调研并设计算法优化
\begin{itemize}
  \item 连贯性影响阅读/听体验及语义正确性,是抽取式摘要难点;应用基于REINFORCE的End2End方案,效果不实用;设计、实现基于策略的MMR调整法,效果略正向,合入线上摘要系统
  \item 生成式摘要标注成本高,提升Low-Resource下摘要生成有必要性;完成半监督摘要方法调研、验证预训练(如BART)在少数据量下的有效性;正在尝试利用标题远监督信息提升重要信息覆盖
\end{itemize}
\end{onehalfspacing}

\section{\faGraduationCap\ 教育背景}
\datedsubsection{\textbf{哈尔滨工业大学}, 哈尔滨, 黑龙江}{2015年9月 -- 2017年7月}
\textit{硕士研究生}\ 计算机科学与技术 \hfill 社会计算与信息检索研究中心(导师:车万翔\ 教授)
\datedsubsection{\textbf{哈尔滨工业大学}, 哈尔滨, 黑龙江}{2011年8月 -- 2015年7月}
\textit{学士}\ 计算机科学与技术 \hfill 学分绩排名前10\%


\section{\faCogs\ 专业技能 \& 其他}
% increase linespacing [parsep=0.5ex]
\begin{onehalfspacing} 
\begin{itemize}
  \item \makebox[4.2em][s]{研究方面}:  \makebox[34em][s]{拥有机器学习、深度学习基本素养;有文本摘要、词法分析、文本分类经验}
  \item \makebox[4.2em][s]{工程方面}:  \makebox[34em][s]{具备计算机思维;熟悉C++11/Python/Shell; 有PyTorch/TF1.x/飞桨开发经验}
  \item \makebox[4.2em][s]{个人评价}:  分析解决问题能力中上,开发能力中上,学术研究能力提升中;尊重排期 \& 精益求精
  \item \makebox[4.2em][s]{\ G i t H u b \ }: https://github.com/memeda
\end{itemize}  
\end{onehalfspacing}
  
% Reference Test
%\datedsubsection{\textbf{Paper Title\cite{zaharia2012resilient}}}{May. 2015}
%An xxx optimized for xxx\cite{verma2015large}
%\begin{itemize}
%  \item main contribution
%\end{itemize}

%% Reference
%\newpage
%\bibliographystyle{IEEETran}
%\bibliography{mycite}
\end{document}
