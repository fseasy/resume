% !TEX TS-program = xelatex
% !TEX encoding = UTF-8 Unicode
% !Mode:: "TeX:UTF-8"

\documentclass{resume}
\usepackage{zh_CN-Adobefonts_external} % Simplified Chinese Support using external fonts (./fonts/zh_CN-Adobe/)
%\usepackage{zh_CN-Adobefonts_internal} % Simplified Chinese Support using system fonts
\usepackage{linespacing_fix} % disable extra space before next section
\usepackage{cite}

\begin{document}
\pagenumbering{gobble} % suppress displaying page number

\name{徐伟}

% {E-mail}{mobilephone}{homepage}
% be careful of _ in emaill address
\contactInfo{readonlyfile@hotmail.com}{(+86) 180-0451-xxxx}{https://memeda.github.io}
% {E-mail}{mobilephone}
% keep the last empty braces!
%\contactInfo{xxx@yuanbin.me}{(+86) 131-221-87xxx}{}

\section{\faGraduationCap\  教育背景}
\datedsubsection{\textbf{哈尔滨工业大学}, 哈尔滨,黑龙江}{2015年9月 -- 至今}
\textit{在读硕士研究生}\ 计算机科学与技术, 预计 2017 年 7 月毕业
\datedsubsection{\textbf{哈尔滨工业大学}, 哈尔滨, 黑龙江}{2011年8月 -- 2015年7月}
\textit{学士}\ 计算机科学与技术 , 学分绩排名前10\%

\section{\faUsers\ 实习/项目经历}
\datedsubsection{\textbf{百度},北京}{2014年6月 -- 2014年9月}
\role{C/C++ , Python , shell , MapReduce}{自然语言处理部门 实习}
知识库抽取程序在Hadoop集群的部署。
\begin{itemize}
  \item 验证核心程序正确性;提升脚本处理性能。
  \item 在5个集群上部署抽取处理程序,在有限机器上持续处理20K个百灵库。
  \item 历时一个月,抽取200+TB知识。
\end{itemize}

\datedsubsection{\textbf{面向不均匀类别的文本分类系统设计与实现}}{2015年3月 -- 2015年6月}
\role{Python}{本科毕业设计}
\begin{onehalfspacing}
最终我们构建了两个系统,一个泛化能力更强,一个经仔细调参后效果可能更佳。
\begin{itemize}
  \item 在中英文均衡和不均衡数据集上,通过组合不同特征、分类器等做了大量的对比实验。
  \item 调研尽可能多的方法,并做实验比较在不均衡数据集上应用各种方法的效果。
\end{itemize}
\end{onehalfspacing}

\datedsubsection{\textbf{饮食地图}}{2014年10月 -- 至今}
\role{Python , MySQL}{系统拓展、维护}
\begin{onehalfspacing}
在后台从微博语料中识别饮食行为并计算各种饮食在不同地区、时间等条件下的分布;
在前端展示分析结果, http://ys.8wss.com/ .
\begin{itemize}
  \item 快速熟悉前端项目,修复BUG.
  \item 扩展后台功能,修改和部署抽取程序。
\end{itemize}
\end{onehalfspacing}

% Reference Test
%\datedsubsection{\textbf{Paper Title\cite{zaharia2012resilient}}}{May. 2015}
%An xxx optimized for xxx\cite{verma2015large}
%\begin{itemize}
%  \item main contribution
%\end{itemize}

\section{\faCogs\ IT 技能}
% increase linespacing [parsep=0.5ex]
\begin{itemize}[parsep=0.5ex]
  \item 编程语言: 偏好C/C++,熟练使用Python,有Shell,JS ,Java基础。
  \item 工程经验: 熟悉部分脚本和WEB编写,做过一点QT、MFC、Java应用。
  \item NLP领域经验:熟悉文本分类、分词,了解一点使用神经网络做序列标注的知识。
\end{itemize}

\section{\faHeartO\ 获奖情况}
\datedline{国家奖学金}{2014年10月}

\section{\faInfo\ 其他}
% increase linespacing [parsep=0.5ex]
\begin{itemize}[parsep=0.5ex]
  \item 技术博客: http://blog.csdn.net/readonlyfile
  \item GitHub: https://github.com/memeda
  \item 外语: 英语6级
\end{itemize}

%% Reference
%\newpage
%\bibliographystyle{IEEETran}
%\bibliography{mycite}
\end{document}
